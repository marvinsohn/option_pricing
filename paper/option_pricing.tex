\documentclass[11pt, a4paper, leqno]{article}
\usepackage{a4wide}
\usepackage[T1]{fontenc}
\usepackage[utf8]{inputenc}
\usepackage{float, afterpage, rotating, graphicx}
\usepackage{epstopdf}
\usepackage{longtable, booktabs, tabularx}
\usepackage{fancyvrb, moreverb, relsize}
\usepackage{eurosym, calc}
% \usepackage{chngcntr}
\usepackage{amsmath, amssymb, amsfonts, amsthm, bm}
\usepackage{caption}
\usepackage{mdwlist}
\usepackage{xfrac}
\usepackage{setspace}
\usepackage[dvipsnames]{xcolor}
\usepackage{subcaption}
\usepackage{minibox}
% \usepackage{pdf14} % Enable for Manuscriptcentral -- can't handle pdf 1.5
% \usepackage{endfloat} % Enable to move tables / figures to the end. Useful for some
% submissions.

\usepackage[
    natbib=true,
    bibencoding=inputenc,
    bibstyle=authoryear-ibid,
    citestyle=authoryear-comp,
    maxcitenames=3,
    maxbibnames=10,
    useprefix=false,
    sortcites=true,
    backend=bibtex
]{biblatex}
\AtBeginDocument{\toggletrue{blx@useprefix}}
\AtBeginBibliography{\togglefalse{blx@useprefix}}
\setlength{\bibitemsep}{1.5ex}
\addbibresource{refs.bib}

\usepackage[unicode=true]{hyperref}
\hypersetup{
    colorlinks=true,
    linkcolor=black,
    anchorcolor=black,
    citecolor=NavyBlue,
    filecolor=black,
    menucolor=black,
    runcolor=black,
    urlcolor=NavyBlue
}


\widowpenalty=10000
\clubpenalty=10000

\setlength{\parskip}{1ex}
\setlength{\parindent}{0ex}
\setstretch{1.5}


\begin{document}

\title{Variance Reduction Techniques of Monte Carlo Simulations for Option Pricing\thanks{Marvin Sohn, University of Bonn. Email: \href{mailto:marvin.sohn@uni-bonn.de}{\nolinkurl{marvin [dot] sohn [at] uni-bonn [dot] de}}.}}

\author{Marvin Sohn}

\date{
    {\bf Preliminary -- please do not quote}
    \\[1ex]
    \today
}

\maketitle


\begin{abstract}
    This project aims to compare and evaluate various variance reduction techniques used in Monte Carlo simulations for option pricing.
\end{abstract}

\clearpage


\section{Introduction} % (fold)
\label{sec:introduction}

Monte Carlo simulation is a widely used technique for the pricing of financial derivatives, including European vanilla options. Despite its flexibility and versatility, Monte Carlo simulation can be computationally expensive, requiring a large number of simulations to achieve accurate results. This can be a significant drawback in applications where time and computational resources are limited, such as real-time trading environments.

To address this challenge, researchers have developed a range of variance reduction techniques that aim to reduce the variance of the simulation output while maintaining its accuracy. These techniques include antithetic variates, delta-control variates, gamma-control variates, and stratified sampling, among others.

\section{Results}\label{sec:results}


This was created by \cite{Glassermann:2004}, \cite{Hilpisch:2015} and \cite{Clewlow:1998}.

\begin{table}[!h]
    \input{../bld/table_results.tex}
    \caption{\label{tab:results}\emph{Python:} Results of Monte Carlo Simulations for European Vanilla Option.}
\end{table}


% section introduction (end)



\setstretch{1}
\printbibliography
\setstretch{1.5}


% \appendix

% The chngctr package is needed for the following lines.
% \counterwithin{table}{section}
% \counterwithin{figure}{section}

\end{document}
